% 20190129
% It will be increased by 1 after \chapter
\setcounter{chapter}{18}
% Set this page number
\setcounter{page}{321}


\chapter{Sistem persamaan linier}
Sistem persamaan linier, yang merupakan kumpulan persamaan-persamaan linier, dapat diselesaikan dengan berbagai metode seperti inversi matriks, reduksi baris \footnote{Eric W. Weisstein, "Gaussian Elimination", from MathWorld--A Wolfram Web Resource, url \url{http://mathworld.wolfram.com/GaussianElimination.html} [20160905].}
, dan beberapa metode lainnya. Sistem persamaan linier sendiri merupakan hasil dari upaya untuk memecahkan suatu permasalahan tertentu, seperti arus listrik dalam
rangkaian multi simpul.


%
\section{Persamaan linier}


%
\section{Sistem persamaan linier}


%
\section{Penyelesaian}


%
\section{Pertanyaan}
\begin{enumerate}
\item Buatlah algoritma untuk membahas sistem persamaan linier yang terdiri dari dua persamaan $y_1 = c_{11} x_1 + c_{12} x_2$ dan $y_2 = c_{21} x_1 + c_{22} x_2$ bila telah diketahui nilai-nilai $(x_{1a}, x_{2a}, y_{1a})$ dan $(x_{1a}, x_{2a}, y_{2a})$ untuk masing-masing persamaan. Tunjukkan rumus untuk memperoleh $c_{11}$, $c_{12}$, $c_{21}$, dan $c_{22}$.
\end{enumerate}
